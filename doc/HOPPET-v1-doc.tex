\documentclass[12pt]{article}
%%  TO DO:
%%
%%  Make sure hopper is replaced with final name...
%
% Possible names:
%  HOAXE: High-Order Accelerated X-space Evolution
%
\usepackage{a4wide}
\usepackage{amsmath}
\usepackage{amssymb}
\usepackage{url}
\usepackage{xspace}

\newcommand{\bq}{\boldsymbol{q}}
\newcommand{\GeV}{\;\mathrm{GeV}}
\newcommand{\as}{\alpha_s}
\newcommand{\comment}[1]{\textbf{[#1]}}
\newcommand{\eg}{e.g.\ }
\newcommand{\ie}{i.e.\ }
\newcommand{\MSbar}{\overline{\mathrm{MS}}}
\newcommand{\hoppet}{\textsc{hoppet}\xspace}
\newcommand{\ttt}[1]{\texttt{#1}}
\newcommand{\order}[1]{{\cal O}\left(#1\right)}

%-----------------------------------------
\title{Higher Order Perturbative Parton Evolution Toolkit \\
  a.k.a.\ HOPPETv1
}
\author{G.~P. Salam\\
  LPTHE, Univerities of Paris 6 \& 7 and CNRS, Paris 75005, France.
} 
\date{}


%======================================================================
\begin{document}

\maketitle 

\abstract{This document describes a Fortran~95 package for carrying
  out DGLAP evolution and other common manipulations of parton
  distribution functions (PDFs). The PDFs are represented on a grid in
  $x$-space so as to avoid limitations on the functional form of input
  distributions.  Good speed and accuracy are obtained through the
  representation of splitting functions in terms of their convolution
  with a set of piecewise polynomial basis functions, and Runge-Kutta
  techniques are used for the evolution in $Q$.  Unpolarised evolution
  is provided to NNLO, including heavy-quark thresholds \comment{say
    MSbar?}, and polarised evolution to NLO. The code is structured so
  as to provide simple access to the objects representing splitting
  function and PDFs, making it possible for a user to extend the
  facilities already provided.
  %
  A `vanilla' interface is also available \comment{in progress},
  facilitating use of the evolution part of the code from f77, C and
  C++.  \smallskip
  
  \textbf{Note:} this document describes version~1 of the package.  A
  far more object-oriented version (v2) has undergone substantial
  development, making use of the so-called TR15581 part of
  Fortran~2003 (allocatable arrays as part of derived types).  While
  v2 (available on request) is more flexible and structured than v1,
  it is also considerably larger, not quite as fast, less tolerant of
  all but the most recent compilers, and currently lacking the
  tabulation feature. For this reason it was deemed worthwhile to
  provide a documented version of v1.}

\newpage

\tableofcontents

%======================================================================
%======================================================================
\section{Introduction}
\label{sec:intro}

There has been considerable discussion over the past years~\cite{}, of
numerical solutions of the DGLAP equation~\cite{DGLAP} for the
evolution QCD parton distributions.


The Dokshitzer-Gribov-Lipatov-Altarelli-Parisi (DGLAP) equation
\cite{DGLAP} is a renormalisation group equation for the quantity
$q_i(x,Q^2)$, the density of partons of type (or flavour) $i$ carrying
a fraction $x$ of the longitudinal momentum of a hadron, when resolved
at a scale $Q$. It is one of the fundamental equations of perturbative
quantum chromodynamics (QCD), being central to all theoretical
predictions for hadron-hadron and lepton-hadron colliders.

Technically, it is a matrix integro-differential equation,
\begin{equation}
  \label{eq:dglap}
  \frac{\partial q_i(x,Q^2)}{\partial \ln Q^2} = \int \frac{dz}{z}
  P_{ij}(z,Q^2) q_j(x,Q^2)\,,
\end{equation}
whose kernel elements $P_{ij}(z,Q^2)$ are known as splitting
functions, since they describe the splitting of a parton of kind $j$
into a parton of kind $i$ carrying a fraction $z$ of the longitudinal
momentum of $j$. The parton densities themselves $q_i(x,Q^2)$ are
essentially non-perturbative, since they depend on physics at hadronic
mass scales $\lesssim 1 \GeV$, where the QCD coupling is large. On the
other hand the splitting functions are given by a perturbative
expansion in the QCD coupling $\as(Q^2)$. Thus given partial
experimental information on the parton densities\footnote{Actually it
  is not the parton densities, but rather structure functions, which
  can be derived from them perturbatively, that are measured
  experimentally.} %
--- for example over a limited range of $Q$, or for only a subset of
parton flavours --- the DGLAP equations can be used to reconstruct the
parton densities over the full range of $Q$ and for all flavours.

The pivotal role played by the DGLAP equation has motivated a
considerable body of literature discussing its numerical solution
\cite{AllDGLAP}. There exist two main classes of approach: those that
solve the equation directly in $x$-space and those that solve it for
Mellin transforms $q_{iN}(Q^2) = \int dx x^N q_i(x,Q^2)$, of the
parton densities and susbequently invert the transform back to
$x$-space. The latter are of interest because the Mellin transform
converts the convolution of eq.(\ref{eq:dglap}) into a multiplication,
resulting in a continuum of independent matrix differential (rather
than integro-differential) equations, one for each value of $N$,
making the evolution very efficient numerically. The drawback of the
method stems however from the need to know the Mellin transforms of
both the splitting functions and the initial conditions and to a
lesser extent subtleties associated with the inverse Mellin transform.

The $x$-space method is in contrast more flexible, since the inputs
are only required in $x$-space; however it is generally considered to
much less efficient numerically, because of the need to carry out the
convolution in eq.(\ref{eq:dglap}).

To understand the question of efficiencies one should analyse the
number of operations needed to carry out the...



Each has advantages
and drawbacks: the $x$-space method provides information over a range
of $x$ values.

%----------------------------------------------------------------------
\section{Tricks}
\label{tricks}

%......................................................................
\subsection{Higher order matrix representation}
\label{sec:highord}

This is higher-order in numerical sense, not perturbative QCD sense.
We have a lower triangular matrix.

Grid points $y_\alpha$, indicated by index $\alpha$; uniform grid
spacing: $y_\alpha = \alpha \delta y$.
\begin{equation}
  q(y,t) = w_\alpha(y) q_\alpha(t)\,.
\end{equation}
Weights serve to interpolate: $q_\alpha = q(y_\alpha)$,
$w_\alpha(y_\beta) = \delta_{\alpha\beta}$.
\begin{equation}
  (P \otimes q)_\alpha = 
\end{equation}


%......................................................................
\subsection{Evolution operators}
\label{sec:evop}
Have
\begin{equation}
  \partial_t q_{\alpha}(t)  = \sum_{\beta} P_{\alpha\beta}(t) q_\beta(t)
\end{equation}
Introduce $M_{\alpha\beta}(t_0) = \delta_{\alpha\beta}$. Solve 
\begin{equation}
  \partial_t M_{\alpha\beta}(t)  = \sum_{\gamma} P_{\alpha\gamma}(t)
  M_{\gamma\beta}(t)
\end{equation}
Then 
\begin{equation}
  q_{\alpha}(t)  = \sum_{\beta} M_{\alpha\beta}(t) q_\beta(t_0)
\end{equation}
If once can rewrite $P$ as $P_{\alpha\beta} = \cal P_{\alpha-\beta}$,
then similarly one can rewrite $M_{\alpha\beta} = \cal
M_{\alpha-\beta}$, and it is as simple to determine
$M_{\alpha\beta}(t)$ as it is to determine the evolution of a single
vector $q_\alpha$, i.e.\ one just evolves a single column, $\beta =
0$, of $M_{\alpha\beta}(t)$.


%----------------------------------------------------------------------
\section{Single-flavour grids and convolutions}

The software package is written in f90. This has considerable
advantages compared to f77, as will be seen in the discussion of the
program, though it does lack a number of fully object-oriented
features and this sometimes restricts the scope for expressiveness.
Fortran~90 perhaps not being the best known language in the
high-energy community, occasionally some indications will be give to
help the reader with less-known language constructs, with further
information in appendix~\ref{sec:f95appendix}.

All routines described in this section need access to the
\texttt{convolution} module, which can either be obtained directly by
adding a 
\begin{verbatim}
  use convolution
\end{verbatim}
statement at the beginning of the relevant subprogram (before
any \ttt{implicit none} or similar declarations). Alternatively, as
with the rest of the routines described in this documentation, it can
be accessed indirectly through the \ttt{hoppet\_v1} module
\begin{verbatim}
  use hoppet_v1
\end{verbatim}
Unless you are delving into the innards of the \hoppet, the latter is
more convenient since it provides access to everything you are likely
to need.

%......................................................................
\subsection{Grid Definitions (\texttt{grid\_def})}
\label{sec:grid}

The grid is the central element of the PDF evolution. Information
concerning the grid is stored in a derived type \texttt{grid\_def},
which can be initialised as follows:
\begin{verbatim}
  type(grid_def) :: grid
  call InitGridDef(grid,dy=0.1_dp,ymax=10.0_dp,order=3)
\end{verbatim}
This initialises a grid between $x=1$ and down to $x =
e^{-\texttt{ymax}}$, with grid spacing in $y = \ln 1/x$ of
\texttt{dy}. The grid will use 3rd order interpolation.

One notes the use of keyword arguments --- the keywords are not
mandatory in this case, but have been included to improve the
legibility. 

Having defined a grid, the user need not worry about the details of
the grid representation.

In line with the convention set out in the Fortran~90 edition of
Numerical Recipes \cite{NRf90} we shall use \texttt{\_dp} to indicate
that numbers are in double precision, and \ttt{real(dp)} to declare
double precision variables. The integer parameter \ttt{dp} is defined
in the \texttt{module types} (and available indirectly through
\ttt{module hoppet\_v1}).

As discussed above, it is often useful to have multiple grids, with
coarse coverage at small $x$ and fine coverage at high $x$. To support
this we can first defined an array of sub-grids, and then use it
initialise a combined grid
\begin{verbatim}
  type(grid_def) :: grid, subgrids(3)

  ! define the various sub-grids
  call InitGridDef(subgrids(1),dy=0.30_dp, ymax=10.0_dp, order=3)
  call InitGridDef(subgrids(2),dy=0.10_dp, ymax= 2.0_dp, order=3)
  call InitGridDef(subgrids(3),dy=0.03333_dp, ymax= 0.6_dp, order=3)

  ! put them together into a single combined grid
  call InitGridDef(grid, subgrids, locked=.true.)
\end{verbatim}
When combining them the \ttt{locked=.true.} option has been specified,
which ensures that after any convolution, information from the finer
grids is propagated into the coarser ones. This places some
requirements on the grid spacings, notably that a coarse grid have a
spacing that is a multiple of that of the next finest grid. If the
requirements are not satisfied by the \ttt{subgrids} that have been
provided, then new similar, but more suitable subgrids are
automatically generated. 

Note that the two kinds of invocation of \ttt{InitGridDef} actually
correspond to different subroutines. The fortran~90 compiler
automatically selects the correct one on the basis of the types of
arguments passed.

Though only grids that are uniform in $y$ have been implemented,
nearly all of the description that follows and all code outside the
\texttt{convolution} module are independent of this detail (the only
exception being certain statements about timings). Therefore were
there to be a strong motivation for an alternative, non-uniform grid,
it would suffice to modify the \texttt{convolution} module, while the
rest of the library (and its interfaces) would remain unchanged.

%......................................................................
\subsection{$x$-space functions}
\label{sec:xspc}

Here we encounter our first difficulties, and departure from an
object-oriented `ideal'. Normal $x$-space functions (such as PDFs) are
held in double precision arrays, which are to be allocated as follows
\begin{verbatim}
  real(dp), pointer :: gluon(:)
  call AllocGridQuant(grid,gluon)
\end{verbatim}
Note that for this to work, \texttt{gluon(:)} should be a
\texttt{pointer} not just have the \texttt{allocatable} attribute. To
deallocate a grid quantity, one may safely use the f90
\texttt{deallocate} command.

Since \texttt{gluon(:)} is just an array, it carries no information
about the \texttt{grid}. Therefore to set and access its value, one
must always provide the information about the \texttt{grid}. This is
not entirely satisfactory, and is one of the drawbacks of the use of
f90, as will be explained later on.

There are a number of ways of setting a grid quantity. Suppose we have
a subroutine
\begin{verbatim}
  subroutine example_gluon(y,g)
    use types                !! defines "dp" (double precision) kind
    implicit none
    real(dp), intent(in)  :: y
    real(dp), intent(out) :: g
    real(dp) :: x
    
    x = exp(-y)
    g = 1.7_dp * x**(-0.1_dp) * (1-x)**5 
  end subroutine example_gluon
\end{verbatim}
Then we can call
\begin{verbatim}
  call InitGridQuantSub(grid,gluon,example_gluon)
\end{verbatim}
to initialise \texttt{gluon} with a representation of the return value
from the subroutine \texttt{example\_gluon}. An alternative way is to
make use of functions \texttt{xValues} or \texttt{yValues} that
respectively return the $x$ or $y$ values of all points on the grid:
\begin{verbatim}
  real(dp), poitner :: gluon,xvals
  call AllocGridQuant(grid,gluon)
  call AllocGridQuant(grid,xvals)
  xvals = xValues(grid)
  gluon = 1.7_dp * xvals**(-0.1_dp) * (1-xvals)**5 
  deallocate(xvals)
\end{verbatim}
Though more laborious insofar as one has to worry about some extra
allocation and deallocation, it has the advantage that one no longer
has to write a separate subroutine.
 

To then access the gluon at a given value of $y = \ln 1/x$, one
proceeds as follows
\begin{verbatim}
  real(dp) :: y, gluon_at_y
  ...
  y = 5.0_dp
  gluon_at_y = EvalGridQuant(grid,gluon,y)
\end{verbatim}
We have to supply the \texttt{grid} argument to \ttt{EvalGridQuant}
because the \ttt{gluon} array itself carries no information about the
grid (other than its size).

A less efficient, but perhaps more `object-oriented' way of accessing
the gluon is via the notation \comment{make it more efficient and
  safe: get \texttt{y.with.grid} to return a structure directly
  containing weights? Should see how inefficient it really is...}
\begin{verbatim}
  gluon_at_y = gluon .aty. (y.with.grid)
\end{verbatim}
There also exists an \ttt{.atx.} operator for evaluating the PDF at a
given $x$ value.  Many of these procedures and operators are
overloaded so as to work with higher-dimensional arrays of grid
quantities, for example a \texttt{pdf(:,:)}. The first index will
always correspond to the representation on the grid, while the second
index would here indicate the flavour.

Note that arithmetic operators all have higher precedence than
library-defined operators such as \texttt{.aty.}; accordingly some
ways of writing things are more efficient than others:
\begin{verbatim}
  gluon_at_y = 2 * gluon .aty. (y.with.grid)   ! very inefficient
  gluon_at_y = 2 * (gluon .aty. (y.with.grid)) ! fairly efficient
  gluon_at_y = 2 * EvalGridQuant(grid,gluon,y) ! most efficient
\end{verbatim}
In the first case the whole of the array \texttt{gluon} is multiplied
by 2, and then the result is evaluated at $y$, whereas in the second
case only the result of the gluon at $y$ is multiplied by 2.

%......................................................................
\subsection{Grid Convolution operators}
\label{sec:conv}

While it is relatively straightforward internally to represent a
grid-quantity (e.g.\ parton distribution) as an array, for convolution
operators, it is generally useful to have certain extra
information. Accordingly a derived type has been defined to hold a
convolution operator, and routines are provided for allocation and
initialisation of splitting functions.
\begin{verbatim}
  type(grid_conv) :: Pgg
  call AllocGridConv(grid,Pgg)
  call InitGridConv(Pgg, Pgg_func)
\end{verbatim}
where the $P_{gg}$ splitting function is provided in the form of the
function \texttt{Pgg\_func}
\begin{verbatim}
  ! returns various components of exp(-y) P_gg (exp(-y))
  real(dp) function Pgg_func(y)
    use types
    use convolution_communicator ! provides cc_piece, and cc_REAL,...
    use qcd                      ! provides CA, TR, nf, ...
    implicit none
    real(dp), intent(in) :: y
    real(dp)             :: x

    x = exp(-y); Pgg_func = zero
    if (cc_piece == cc_DELTA) then 
       Pgg_func = (11*CA - 4*nf*TR)/6.0_dp
    else
       if (cc_piece == cc_REAL .or. cc_piece == cc_REALVIRT) & 
           & Pgg_func = 2*CA*(x/(one-x) + (one-x)/x + x*(1-x))
       if (cc_piece == cc_VIRT .or. cc_piece == cc_REALVIRT) & 
           & Pgg_func = Pgg_func - 2*CA*one/(one-x)
       Pgg_func = Pgg_func * x       ! remember to return x * Pgg
    end if
  end function Pgg_func
\end{verbatim}
To address the issue that convolution operators can involve
plus-distributions and delta functions, the module
\texttt{convolution\_communicator} contains a variable
\texttt{cc\_piece} which indicates which part of the splitting
function is to be returned --- the real, virtual, real + virtual, or
$\delta$-function pieces. Note that in all cases $P_{gg}(x)$ is to be
returned multiplied by $x$.

The initialisation of a \texttt{grid\_conv} object uses an adaptive
Gaussian integrator (a variant of CERNLIB's \texttt{dgauss}) to
calculate the convolution of the splitting function with trial weight
functions. The default accuracy for these integrations is $10^{-7}$.
It can be modified to value \texttt{eps} with the following subroutine
call
\begin{verbatim}
  call SetDefaultConvolutionEps(eps)
\end{verbatim}
which is to be made before creating the \ttt{grid\_def}
object.\footnote{Alternatively, an optional \ttt{eps} argument can be
  included in the call to \ttt{InitGridDef}.}  Note that this is just
one of the parameters affecting the final accuracy of convolutions. In
practice (unless going to extremely high accuracies) the grid spacing
and interpolation scheme are more critical.

Having allocated and initialised a \texttt{Pgg} splitting function, we
can go on to use it. For example 
\begin{verbatim}
  real(dp), pointer :: Pgg_x_gluon(:)
  ...
  call AllocGridQuant(grid,Pgg_x_gluon)
  Pgg_x_gluon = Pgg .conv. gluon
\end{verbatim}  
Since the return value of \texttt{Pgg .conv.\ gluon} is just an f90
array, one can also write more complex expressions. Supposing we had
defined also a \texttt{Pgq} splitting function and a singlet
\texttt{quark} distribution, as well as $\texttt{as2pi} = \as/2\pi$,
then to first order in $\as$ we could write the gluon evolution
through a step \texttt{dt} in $\ln Q^2$ as
\begin{verbatim}
  gluon = gluon + (as2pi*dt) * ((Pgg .conv. gluon) + (Pgq .conv. quark))
\end{verbatim}
Note that like \texttt{.aty.}, \texttt{.conv.} has a low precedence,
so the use of brackets is important to ensure that the above
expressions are sensible. Alternatively, the issues of precedence can
be addressed by using \texttt{*} (also defined as convolution when it
appears between a splitting function and a PDF) instead of
\texttt{.conv.}:
\begin{verbatim}
  gluon = gluon + (as2pi*dt) * (Pgg*gluon + Pgq*quark)
\end{verbatim}


%......................................................................
\subsubsection{Other operations on \texttt{grid\_conv} objects}
\label{sec:other_grid_conv_ops}
%
It is marginally less transparent to manipulate \texttt{grid\_conv} types
than PDF distributions, but still fairly simple:
\begin{verbatim}
  call AllocGridConv(grid,Pab)         ! Pab memory allocated 
  call InitGridConv(grid,Pab)          ! Pab = 0             (opt.alloc)
  call InitGridConv(Pab,Pcd[,factor])  ! Pab = Pcd [*factor] (opt.alloc)
  call InitGridConv(grid,Pab,function) ! Pab = function      (opt.alloc)

  call SetToZero(Pab)                  ! Pab = 0
  call Multiply (Pab,factor)           ! Pab = Pab * factor
  call AddWithCoeff(Pab,Pcd[,coeff])   ! Pab = Pab + Pcd [*coeff]
  call AddWithCoeff(Pab,function)      ! Pab = Pab + function

  call SetToConvolution(Pab,Pac,Pcb)   ! Pab = Pac.conv.Pcb  (opt.alloc)
  call SetToConvolution(P(:,:),Pa(:,:),Pb(:,:))            ! (opt.alloc)
                                       ! P(:,:) = matmul(Pa(:,:),Pb(:,:))
  call SetToCommutator(P(:,:),Pa(:,:),Pb(:,:))             ! (opt.alloc)
                                       ! P(:,:) = matmul(Pa(:,:),Pb(:,:))
                                       !         -matmul(Pb(:,:),Pa(:,:))    

  call Delete(Pab)                     ! Pab memory freed
\end{verbatim}
Routines labelled ``\texttt{(opt.alloc.)}'' allocate the memory for
the \texttt{grid\_conv} object if the memory has not already been
allocated. (If it has already been allocated it is assumed to
correspond to the same grid as any other \texttt{grid\_conv} objects
in the same subroutine call). Some calls require that one specify the
grid definition being used (\texttt{grid}), because otherwise there is
no way for the subroutine to deduce which grid is being used.

When creating a \texttt{grid\_conv} object for temporary use, it is
important to remember to \texttt{Delete} it afterwards, so as to avoid
memory leaks.

Nearly all the routines are partially overloaded so as to be able to
deal with one and two-dimensional arrays of \texttt{grid\_conv}
objects as well. The exceptions are those that initialise the
\texttt{grid\_conv} object from a function (arrays of functions do not
exist), as well as the convolution routines (for which the extension
to arrays might be considered non-obvious) and the commutation routine
which only has sense for matrices of \texttt{grid\_conv} objects.

%......................................................................
\subsubsection{Derived \texttt{grid\_conv} objects}
\label{sec:derived_grid_conv}

Sometimes it can be cumbersome to manipulate the \texttt{grid\_conv}
objects directly, for example when trying to create a
\texttt{grid\_conv} that represents not a fixed order splitting
function, but the resummed evolution from one scale to another. For
such situations the following approach can be used
\begin{verbatim}
  real(dp), pointer :: probes(:,:)
  type(grid_conv)   :: Pqg, Pgq, Presult
  integer           :: i

  call GetDerivedProbes(grid,probes) ! get the probes
  do i = 1, size(probes,dim=2)       ! carry out operations on each of the probes
    probes(:,i) = Pqg*(Pgq*probes(:,i)) - Pgq*(Pqg*probes(:,i))
  end do
  call AllocGridConv(grid,Presult)
  call SetDerivedConv(Presult,probes) ! Presult = [Pqg,Pgq]
\end{verbatim}
Here \texttt{GetDerivedProbes} allocates and sets up an array of probe
parton distributions. Since a single-flavour parton distribution is a
one-dimensional array of \texttt{real(dp)}, the array of probes is a
two-dimensional array of \texttt{real(dp)}, the second dimension
corresponding to the index of the probe. One then carries out whatever
operations one wishes on each of the probes. Finally with the call to
\texttt{SetDerivedConv}, one can reconstruct a \texttt{grid\_conv}
object that corresponds to the set of operations just carried out

Some comments about memory allocation: the probes are automatically
allocated and deallocated; in contrast the call to
\texttt{SetDerivedConv(Presult,probes)} knows nothing about the grid,
so \texttt{Presult} must have been explicitly allocated for a specific
grid beforehand.

A note of caution: when one's grid is made of nested subgrids with the
locking option set to \texttt{.true.}, after a convolution of a
\texttt{grid\_def} object with a parton distribution, the coarser
grids for the parton distribution are supplemented with more accurate
information from the finer grids. When carrying out multiple
convolutions, this happens after each convolution.  There is no way to
emulate this with a single \texttt{grid\_def} object, and the locking
would actually confuse the reconstruction of resulting
\texttt{grid\_def} object. So when the user requests the probes,
locking is temporarily turned off globally and then reestablished
after the derived \texttt{grid\_object} has been contructed. Among
other things this means that acting with a derived
\texttt{grid\_object} will not be fully equivalent to carrying out the
individual operations separately. In particular the accuracy may be
slightly lower (whatever is lost due to the absence of intermediate
locking).






%======================================================================
%======================================================================
\section{Multi-flavour grids and convolutions}
\label{sec:dglapstructs}

The above discussion holds for any kind of problem involving
convolutions, even if the examples were given in the context of DGLAP
evolution. In this section we shall examine the tools made available
specifically to address the DGLAP evolution problem.


%----------------------------------------------------------------------
\subsection{Full-flavour PDFs and flavour representations}
\label{sec:pdf-objects}

The routines described here are available from the \ttt{pdf\_general}
and \ttt{pdf\_representation} modules, or via the \ttt{hoppet\_v1}
general module.

Full flavour PDFs are just like single flavour PDFs except that they
have an extra dimension. They are represented by arrays, and if you
want \hoppet to deal with allocation for you, they should be pointer
arrays. One can allocate a single PDF (two dimensional
\texttt{real(dp)} array) or an array of PDFs (three-dimensional
\texttt{real(dp)} array)
\begin{verbatim}
  real(dp), pointer :: PDF(:,:), PDFarray(:,:,:)
  call AllocPDF(grid,PDF)            ! allocates PDF(0:,-6:7)
  call AllocPDF(grid,PDFarray,0,10)  ! allocates PDFarray(0:,-6:7,0:10)
\end{verbatim}
The first dimension corresponds to the grid in $y$; the second
dimension corresponds to the flavour index. Its lower bound is $-6$,
as one would expect. What takes a bit more getting used to is that its
upper bound is \textbf{7}. The reason is as follows: the flavour
information can be represented in different ways, for example each
flavour separately, or alternatively as singlet and non-singlet
combinations. In practice both are used inside the program and it is
useful for a PDF distribution to have information about the
representation it is in --- it is this that is stored in
\texttt{PDF(:,7)}.

%......................................................................
\subsubsection{Human representation.}
\label{sec:human-rep}
When a PDF is allocated it is automatically labelled as being in the
\ttt{human} representation, in which $\bar t={-6}, \bar b={-5}, \bar
c={-4}, \bar s={-3}, \bar u={-2}, \bar d={-1}, g={0}, d={1}, u={2},
s={3}, c={4}, b={5}, t={6} $. Constants with names like
\ttt{iflv\_bbar}, \ttt{iflv\_g}, \ttt{iflv\_b}, are defined in
\ttt{module pdf\_representation}, to facilitate symbolic access to the
different flavours.

If you are creating a PDF as an automatic array (one whose bounds are
decided not by the allocation routine, but on the fly), for example in
a function that returns a PDF, then you should label it yourself as
being in the \ttt{human} representation, either with the the
\ttt{LabelPdfAsHuman(pdf)} subroutine call, or by setting
\ttt{pdf(:,7)} to zero:
\begin{verbatim}
module pdf_initial_condition
  use hoppet_v1; implicit none
contains
  function unpolarized_dummy_pdf(xvals) result(pdf)
    real(dp), intent(in) :: xvals(:)
    real(dp)             :: pdf(size(xvals),-6:7)

    ! clean method for labelling as PDF as being in the human representation
    call LabelPdfAsHuman(pdf)
    ! by setting everything to zero (notably pdf(:,7)) representation
    ! is automatically set to be human
    pdf(:,:) = 0
    
    ! iflv_g is pre-defined integer parameter (=0) for symbolic ref. to gluon
    pdf(:,iflv_g) = 1.7_dp * xvals**(-0.1_dp) * (1-xvals)**5
    [... set other flavours here ...]
  end function unpolarized_dummy_pdf
end module pdf_initial_condition
\end{verbatim}
The function has been placed in a module so as to provide an easy way
for a calling routine to have access to its interface (this is needed
for the dimension of \ttt{xvals} to be correctly passed).  Writing a
function such as that above is probably the easiest way of
initialising a pdf:
\begin{verbatim}
  use hoppet_v1; use pdf_initial_condition; implicit none
  type(grid_def)    :: grid
  real(dp), pointer :: pdf(:,:)
  [...]
  call AllocPDF(grid,pdf)
  pdf = unpolarized_dummy_pdf(xValues(grid))
\end{verbatim}
There exist a number of other options, which can be found by browsing
through \ttt{src/pdf\_general.f90}. Of these a sometimes handy one is
\begin{verbatim}
  call AllocPDF(grid,pdf)
  call InitPDF_LHAPDF(grid, pdf, LHAsub, Q)
\end{verbatim}
where \texttt{LHAsub} is the name of a subroutine 
%(to be declared
%\ttt{external} if an explicit interface is not available?) 
with the same interface as LHAPDF's \ttt{evolvePDF} \cite{LHAPDF}:
\begin{verbatim}
  subroutine LHAsub(x,Q,res)
    use types; implicit none
    real(dp), intent(in)  :: x,Q
    real(dp), intent(out) :: res(*) ! on output contains flavours -6:6 at x,Q
    [...]
  end subroutine LHAsub
\end{verbatim}


Having initialised a PDF, to then extract it at a given $y$ value, one
can either examine a particular flavour using the methods described in
section~\ref{sec:xspc}
\begin{verbatim}
  real(dp) :: y, gluon_at_y
  gluon_at_y = pdf(:,iflv_g) .aty. (y.with.grid)
  ! OR
  gluon_at_y = 2 * EvalGridQuant(grid,pdf(:,iflv_g),y) 
\end{verbatim}
or one can extract all flavours simultaneously
\begin{verbatim}
  real(dp) :: pdf_at_y(-6:6)
  pdf_at_y = pdf(:,-6:6) .aty. (y.with.grid)
  ! OR
  pdf_at_y = 2 * EvalGridQuant(grid,pdf(:,-6:6),y) 
\end{verbatim}
with the latter being more efficient if one needs to extract all
flavours simultaneously. Note that here we have explicitly specified
the flavours, here \ttt{-6:6}, that we want.\footnote{If instead we had
  said \ttt{pdf(:,:)} the result would have corresponded to a slice of
  flavours \ttt{-6:7}, \ie including an interpolation of the
  representation labelling information, which would be meaningless.}


%......................................................................
\subsubsection{Evolution representation.} 
\label{sec:evln-rep}
%
For the purpose of carrying out convolutions the \ttt{human}
representation is not very advantageous because the splitting matrix
in flavour space is quite complicated. Accordingly \hoppet uses a
different representation of the flavour internally when carrying out
convolution of splitting matrices with PDFs. For most purposes the
user need not be aware of this. The two exceptions are when a user
plans to create derived splitting matrices (being careless about the
flavour representation will lead to mistakes) or wishes to carry out
repeated convolutions for a fixed $n_f$ value (appropriate manual
changes of the flavour representation can speed things up).


The splitting matrix can be simplied considerably (made diagonal
except for a $2\times2$ singlet block) by switching to a different
flavour representation for \ttt{pdf(:,i)}, as explained in detail in
\cite{vanNeerven:1999ca,vanNeerven:2000uj}
\begin{equation}
  \label{eq:diag_split}
  \begin{array}{r | l| l }
     i & \mbox{name} & q_i \\ \hline
      -6\ldots-(n_f+1) & q_i & q_i\\
     -n_f\ldots -2 & q_{\mathrm{NS},i}^{-} & (q_i -  {\bar q}_i) - (q_1 - {\bar q}_1)\\
      -1           & q_{\mathrm{NS}}^{V} & \sum_{j=1}^{n_f} (q_j -  {\bar q}_j)\\
       0           & g & \textrm{gluon} \\
       1           & \Sigma & \sum_{j=1}^{n_f} (q_j +  {\bar q}_j)\\
     2\ldots n_f & q_{\mathrm{NS},i}^{+} & (q_i +  {\bar q}_i) - (q_1 + {\bar q}_1)\\
      (n_f+1)\ldots6 & q_i & q_i
  \end{array}
\end{equation}
When carrying out a convolution, the only non-diagonal part is the
block containing indices $0,1$. This representation is referred to as
the \ttt{evln} representation. Whereas the \ttt{human} representation
is $n_f$-independent the \ttt{evln} depends on $n_f$ through the
$\Sigma$ and $q_{\mathrm{NS}}^{V}$ entries and the fact that flavours
beyond $n_f$ are left in the human representation (since they are
inactive for evolution with $n_f$ flavours).

To take a PDF in the \ttt{human} representation and make a copy in an
\ttt{evln} representation, one uses the \ttt{CopyHumanPdfToEvln} routine
\begin{verbatim}
  real(dp), pointer :: pdf_human(:,:), pdf_evln(:,:)
  integer           :: nf_lcl

  [... setting up pdf_human, nf_lcl, etc. ...] 
  call AllocPDF(grid,pdf_evln)  ! or it might be an automatic array
  call CopyHumanPdfToEvln(nf_lcl, pdf_human, pdf_evln)
\end{verbatim}
where one specifies the $n_f$ value for the \ttt{evln} representation.
One can go in the opposite direction with
\begin{verbatim}
  call CopyEvlnPdfToHuman(nf_lcl, pdf_evln, pdf_human)
\end{verbatim}


%----------------------------------------------------------------------
\subsection{Splitting \comment{+coeff.fn} function matrices}
\label{sec:splitt-funct-matr}

Splitting function matrices and their actions on PDFs are defined in
\ttt{module dglap\_objects} (accessible as usual from \ttt{module
  hoppet\_v1}). They have type \ttt{split\_mat}. Below we shall discuss
routines for creating specific predefined DGLAP splitting matrices,
but for now we consider a general splitting matrix.

Allocation,
\begin{verbatim}
  type(split_mat) :: P
  integer         :: nf_lcl
  call AllocSplitMat(grid, P, nf_lcl)
\end{verbatim}
is similar to that for \ttt{dglap\_objects}. The crucial difference is
that one must supply a value for $n_f$, so that when the splitting
matrix acts on a PDF it knows which flavours are decoupled. From the
point of view of subsequent initialization a \ttt{split\_mat} object
just consists of a set of splitting functions, which act on the
components given in eq.(\ref{eq:diag_split}). If need be, they can be
initialized by hand, for example
\begin{verbatim}
  call InitGridConv(grid,P%qq      , P_function_qq      )
  call InitGridConv(grid,P%qg      , P_function_qg      )
  call InitGridConv(grid,P%gq      , P_function_gq      )
  call InitGridConv(grid,P%gg      , P_function_gg      )
  call InitGridConv(grid,P%NS_plus , P_function_NS_plus )
  call InitGridConv(grid,P%NS_minus, P_function_NS_minus)
  call InitGridConv(grid,P%NS_V    , P_function_NS_V    )
\end{verbatim}
We can then write
\begin{verbatim}
  real(dp), pointer :: q(:,:), delta_q(:,:)
  [... allocations, etc. ...]
  delta_q = P .conv. q
  ! OR
  delta_q = P * q
\end{verbatim}
and $\ttt{delta\_q}$ will have the following components
\begin{align}
  \label{eq:Pmat_on_q}
  \left(\!\!
    \begin{array}{c}
      \delta\Sigma\\
       \delta g
    \end{array}
  \!\!\right)
    \;&= \;
  \left(
    \begin{array}{cc}
      \ttt{P\%qq} & \ttt{P\%qg}\\
      \ttt{P\%gq} & \ttt{P\%gg}
    \end{array}
  \right) 
  \otimes
  \left(\!\!
    \begin{array}{c}
      \Sigma\\
       g
    \end{array}
    \!\!\right) 
  \nonumber\\[3pt]
%
  \delta q^+_{\mathrm{NS},i} \;&=\; \ttt{P\%NS\_plus} \otimes
  q^+_{\mathrm{NS},i}\\[3pt] 
%
  \delta q^-_{\mathrm{NS},i} \;&=\; \ttt{P\%NS\_minus} \otimes
  q^-_{\mathrm{NS},i}\nonumber \\[3pt]
% 
  \delta q^V_{\mathrm{NS}} \;&=\; \ttt{P\%NS\_V} \otimes
  q^V_{\mathrm{NS}} \nonumber
\end{align}
We have written the result in terms of components in the \ttt{evln}
representation (and this is the representation use for the actual
convolutions). When a convolution with a PDF in \ttt{human}
representation is carried out, the program automatically copies the
PDF to the \ttt{evln} representation, carries out the convolution and
converts the result back to the \ttt{human} representation.
%
The cost of changing a representation is $\order{N}$, whereas the
convolution is $\order{N^2}$, so in principle the former is
negligible. In practice, especially when aiming for high speed at low
$N$, the change of representation can imply a significant cost. In
such cases, if multiple convolutions are to be carried out, it may be
advantageous to manually change into the appropriate \ttt{evln}
representation, carry out all the convolutions and then change back to
the \ttt{human} manually representation at the end.
%
\comment{Caveat: currently the internal information in \ttt{q(:,7)}
  does not distinguish between \ttt{evln} representations for
  different $n_f$ values; so if \ttt{q} is in the \ttt{evln}
  representation for the wrong $n_f$ value you will get a wrong
  answer. This is bad and should be fixed!}

As for \ttt{grid\_conv} objects, a variety of routines have been
implemented to help manipulate splitting matrices:
\begin{verbatim}
  type(split_mat) :: PA, PB, PC
  real(dp) :: factor

  -- InitSplitMat(grid, PA) ???
  call InitSplitMat(PA,PB[,factor])   ! PA = PB [*factor]   (opt.alloc)

  call SetToZero(PA)                  ! PA = 0
  call Multiply(PA,factor)            ! PA = PA * factor               
  call AddWithCoeff(PA,PB[,factor])   ! PA = PA + PB [*factor]

  call SetToConvolution(PA,PB,PC)     ! PA = PB*PC          (opt.alloc)
  call SetToCommutator(PA,PB,PC)      ! PA = PB*PC-PC*PB    (opt.alloc)
  
  call Delete(split_mat)                   ! PA's memory freed
\end{verbatim}


%......................................................................
\subsubsection{Derived splitting matrices}
\label{sec:derived-split-matrices}

As with \ttt{grid\_conv} objects, \hoppet provides means to construct
a \ttt{split\_mat} object that corresponds to an arbitrary series of
\ttt{split\_mat} operations, as long as they all involve the same
value of $n_f$. One proceeds in a very similar way,
\begin{verbatim}
  real(dp), pointer :: probes(:,:,:)
  type(split\_mat)  :: PA, PB, Pcomm
  integer           :: i

  [...]
  call GetDerivedSplitMatProbes(grid,probes) ! get the probes
  do i = 1, size(probes,dim=3)               ! carry out operations on each of the probes
    probes(:,i) = PA*(PB*probes(:,:,i)) - PB*(PA*probes(:,:,i))
  end do
  call AllocSplitMat(grid,Pcomm,nf_lcl)      ! provide nf info in initialisation
  call SetDerivedConv(Pcomm,probes)          ! Presult = [Pqg,Pgq]
\end{verbatim}
As in section~\ref{sec:derived_grid_conv}, we first need to set up
some `probe' PDFs (note the extra dimension compared to earlier, since
we also have flavour information; the probe index always corresponds
to the last dimension); then we act on those probes; finally we
allocate the splitting matrix, and set its contents based on the
probes, which are then automatically deallocated.



%----------------------------------------------------------------------
\subsection{The DGLAP convolution components}
\label{sec:dglap_holder}

%......................................................................
\subsubsection{QCD constants}
\label{sec:qcd}

The splitting functions that we set up will depend on various QCD
constants ($n_f$, colour factors), so it is useful to here to
summarize how they are dealt with within the program.

The treatment of the QCD constants is \emph{not} object oriented.
There is a module (\ttt{qcd}) that provides access to commonly used
constants in QCD:
\begin{verbatim}
  real(dp) :: ca, cf, tr, nf
  integer  :: nf_int
  
  real(dp) :: beta0, beta1, beta2
  [ ... ]
\end{verbatim}
Note that \ttt{nf} is in double precision\footnote{The author has
  forgotten why, but suspects there might some good reason and is
  therefore nervous of changing it!} --- if you want the integer value
of $n_f$, use \ttt{nf\_int}. 

To set the value of $n_f$, call
\begin{verbatim}
  integer :: nf_lcl
  call qcd_SetNf(nf_lcl)  
\end{verbatim}
where we've used the variable \ttt{nf\_lcl} to avoid conflicting with
the \ttt{nf} variable provided by the \ttt{qcd} module. Whatever you
do, do not simply modify the value of the \ttt{nf} variable by hand
--- when you call \ttt{qcd\_SetNf} is adjusts a whole set of other
constants (\eg the $\beta$ function coefficients) appropriately.

There are situations in which it's of interest to vary the other
colour factors of QCD --- for that purpose, use
\begin{verbatim}
  real(dp) :: ca_lcl, cf_lcl, tr_lcl
  call qcd_SetGroup(ca_lcl, cf_lcl, tr_lcl)
\end{verbatim}
Again all other constants in the \ttt{qcd} module will be adjusted. A
word of caution: the NNLO splitting functions actually depend on a
colour structure that goes beyond the usual $C_A$, $C_F$ and $T_R$,
namely $d_{abc}d^{abc}$. The means for setting this have yet to be
implemented. \comment{Fix this?}


A comment regarding normalizations: nearly everything perturbative in
\hoppet is defined to be a coefficient of $\as/2\pi$. The one
exception is the $\beta$-function coefficients:
\begin{equation}
  \label{eq:as-ev}
  \frac{d\as}{d\ln Q^2} = -\as (\ttt{beta0}\cdot \as +
  \ttt{beta1}\cdot \as^2 + 
  \ttt{beta2} \cdot\as^3).
\end{equation}
There's no good reason for this. \comment{But now's not the time to
  change it.}

%......................................................................
\subsubsection{DGLAP splitting matrices}
\label{sec:dglap-split}

The module \ttt{dglap\_objects} includes a number of routines for
providing access to the \ttt{split\_mat} objects corresponding to
DGLAP splitting functions
\begin{verbatim}
  type(split_mat) :: P_LO, P_NLO, P_NNLO
  type(split_mat) :: Pp_LO, Pp_NLO        ! polarized

  ! MSbar unpolarized case
  call InitSplitMatLO  (grid, P_LO)
  call InitSplitMatNLO (grid, P_NLO)
  call InitSplitMatNNLO(grid, P_NNLO)

  ! the MSbar polarized case...
  call InitSplitMatPolLO (grid, Pp_LO)
  call InitSplitMatPolNLO(grid, Pp_NLO)
\end{verbatim}
In each case the splitting function is set up for the $n_f$ and
colour-factor values that are current in the $\ttt{qcd}$ module, as
set with the $qcd\_SetNf$ and $qcd\_SetGroup$ subroutine calls. If one
subsequently resets the $n_f$ or color factor values, the $split\_mat$
objects continue to correspond to the $n_f$ and colour factor values
for which they were initially calculated.

With the above subroutines for initializing DGLAP splitting functions,
the normalisation is such that 
\begin{equation}
  \label{eq:dpdf}
  \frac{dq}{d\ln Q^2} \equiv \frac{dq}{dt} = \frac{\as}{2\pi}\,
  \ttt{P\_LO} \otimes q +  
                        \left(\frac{\as}{2\pi}\right)^2 \ttt{P\_NLO}
                        \otimes q + \ldots
\end{equation}
In practice because convolutions take a time $\order{N^2}$ whereas
additions and multiplications take a time $\order{N}$, in a program it
is better to first sum the splitting matrices and then carry out the
convolution,
\begin{verbatim}
  type(split_mat)   :: P_sum
  real(dp), pointer :: q(:,:), dq
  [ ... ]
  call InitSplitMat(P_sum, P_LO)            ! P_sum = P_LO
  call AddWithCoeff(P_sum, P_NLO, as2pi)    ! P_sum = P_sum + as2pi * P_NLO
  dq = (as2pi * dt) * (P_sum .conv. q)
  call Delete(P_sum)
\end{verbatim}
Note the use of brackets in the line setting \ttt{dq}: all scalar
factors are first multiplied together ($\order{1}$) so that we only
have one multiplication of a PDF ($\order{N}$). Note also that we have
chosen to include the \ttt{(as2pi * dt)} factor as multiplying the
pdf, rather than the other option of multiplying {P\_sum}, \ie 
\begin{verbatim}
  call Multiply(P_sum, (as2pi * dt))
  dq =  P_sum .conv. q
\end{verbatim}
The result would have been identical, but splitting matrices with
positive interpolation \ttt{order} essentially amount to an
$\order{7\times \ttt{order} \times N}$ sized array, whereas the PDF is
an $\order{13 N}$ sized array and the for high positive orders that
are sometimes used, it is cheaper to multiply the latter.

A remark concerning NNLO splitting functions: the exact NNLO splitting
functions derived by Moch, Vermaseren and Vogt
\cite{NNLO-NS,NNLO-singlet} involve long (multi-page) expressions in
terms of harmonic polylogarithms of up to weight 4. Very conveniently,
refs.~\cite{NNLO-NS,NNLO-singlet} provide the expressions directly in
terms of fortran code.
%
The harmonic polylogarithms can be evaluated using the \ttt{hplog}
package of Gehrmann and Remiddi \cite{FortranPolyLog}, a copy of which
is included with the \hoppet package. 

The initial integrations needed to create the \ttt{split\_mat} objects
for the exact NNLO splitting functions for the full range of $n_f$
take of the order of minutes.  Since currently there is no option of
storing the splitting matrices in a file, this can be a bit
bothersome. So instead, by default, the program uses the approximate,
parameterized NNLO splitting functions also provided in
\cite{NNLO-NS,NNLO-singlet}. The parameterized splitting functions are
guaranteed to be accurate to within $0.1\%$ --- in practice since they
come in relatively suppressed by two powers of $\as$, the impact on
the evolution tends to be of the order $10^{-5}$ relative effect
\cite{Benchmarks}.

The user can choose whether to obtain the exact or parameterized NNLO
splitting functions using the following calls (to be made before
initialising the splitting matrices)
\begin{verbatim}
  integer :: splitting_variant
  call dglap_Set_nnlo_splitting(splitting_variant)
\end{verbatim}
with the following variants defined (as integer parameters) in the
module \ttt{dglap\_choices}:
\begin{verbatim}
  nnlo_splitting_exact
  nnlo_splitting_param                   [default]
  nnlo_splitting_Nfitav
  nnlo_splitting_Nfiterr1
  nnlo_splitting_Nfiterr2
\end{verbatim}
The last 3 are the parameterizations based on fits to reduced moment
information carried out in \cite{vanNeerven:1999ca,vanNeerven:2000uj}.
Though at the time they represented a valuable (and much used) step on
the way to full NNLO results, nowadays their interest is mainly
historical.

Note that only for the \ttt{nnlo\_splitting\_exact} can the colour
constants be varied (with the caveat about $d^{abc}d_{abc}$).



%......................................................................
\subsubsection{Mass threshold matrices}
\label{sec:mtm}

Still in the \ttt{dglap\_objects} module, we have a type dedicated to
crossing mass thresholds.
\begin{verbatim}
  type(grid_def)           :: grid
  type(mass_threshold_mat) :: MTM_NNLO

  call InitMTMNNLO(grid, MTM_NNLO)   ! MTM_NNLO is coeff of (as/2pi)**2
\end{verbatim}
This is the coefficient, calculated by Buza et al. \cite{NNLO-MTM}
(the authors are thanked for the code corresponding to the
calculation), of $(\as/2\pi)^2$ for the convolution matrix that
accounts for crossing a heavy flavour threshold in $\MSbar$
factorization scheme, at $\mu_F = M_H^\mathrm{pole}$, where
$M_H^\mathrm{pole}$ is the heavy-quark pole mass. Since the
corresponding NLO term is zero, the number of flavours in $\as$ is
immaterial at NNLO (even at NNNLO).

The treatment of $n_f$ in the \ttt{mass\_threshold\_mat} is very
specific because at NNLO, the only order in the $\MSbar$ factorization
scheme at which it's non-zero and known, it is independent of $n_f$.
It's action does of course however depend on $n_f$. Since, as for
\ttt{split\_mat} objects, we don't want to the action of of the
\ttt{mass\_threshold\_mat} to depend on the availability of the
current $n_f$ information from the \ttt{qcd} module, instead we
require that before using a \ttt{mass\_threshold\_mat}, you should
explicitly indicate the number of flavours (defined as including the
new heavy flavour). This is done using a call to the
\ttt{SetNfMTM(MTM\_NNLO,nf\_incl\_heavy)} subroutine. So for example
to take a pdf from 3 flavours to 4 at $\mu_F = M_H^\mathrm{pole}$ one
uses code analogous to the following
\begin{verbatim}
  real(dp) :: pdf_nf3(:,:), pdf_nf4(:,:)
  [ ... ]
  call SetNfMTM(MTM, 4)
  pdf_nf4 = pdf_nf3 + (as2pi)**2 * (MTM_NNLO.conv.pdf_nf3)
\end{verbatim}
The convolution only works if the \ttt{pdf}'s are in the \ttt{human}
representation and an error is given if this is not the case. Any
heavy flavour (charm) present in \ttt{pdf\_nf3} is ignored.

Note that this type is not currently suitable for general changes of
flavour-number. For example if you wish to carry out a change in the
DIS scheme or at a scale $\mu_F \neq M_H^\mathrm{pole}$ then you have
to combine a series of different convolutions (essentially correcting
with the lower number of flavours to the $\MSbar$ factorization scheme
at $\mu_F = M_H^\mathrm{pole}$ before changing the number of flavours
and then correcting back to the original scheme and scale using the
higher number of flavours).


As for the NNLO splitting functions, the mass threshold corrections
come in exact and parameterized variants. By default it is the latter
that is used (provided by Vogt \cite{VogtMTMParam}).  The cost of
initializing with the exact variants of the mass thresholds is much
lower than for the exact NNLO splitting functions (partly because
there is no $n_f$ dependence, partly because it is only one flavour
component of the mass-threshold function that is complex enough to
warrant parameterization). The variant can be chosen by the user
before initialising the \ttt{mass\_threshold\_mat} by making the
following subroutine call:
\begin{verbatim}
  integer :: threshold_variant
  call dglap_Set_nnlo_nfthreshold(threshold_variant)
\end{verbatim}
with the following variants defined (as integer parameters), again in
the module \ttt{dglap\_choices}:
\begin{verbatim}
  nnlo_nfthreshold_exact
  nnlo_nfthreshold_param                 [default]
\end{verbatim}


%......................................................................
\subsubsection{Putting it together: \ttt{dglap\_holder}}

The discussion so far in this subsection was intended to provide the
reader with an overview of the different DGLAP components that have
been implemented and of how they can be initialised individually. This
is useful above all if the user needs to tune the program to some
specific unusual application.

In practice, one foresees that most users will need just a standard
DGLAP evolution framework, and so will prefer not to have to manage
all these components individually. Accordingly \hoppet provides a
type, $\ttt{dglap\_holder}$ which holds all the components needed for
a given kind of evolution. To initialize all information for a
fixed-flavour number evolution, one does as follows
%
\begin{verbatim}
  use hoppet_v1
  type(dglap_holder) :: dglap_h
  integer            :: factscheme, nloop, nf_lcl  

  nloop      = 3                 ! NNLO
  factscheme = factscheme_MSbar  ! or: factscheme_DIS; factscheme_PolMSbar
  call qcd_SetNf(4)              ! set the fixed number of flavours
  ! call qcd_SetGroup(...)       ! if you want different colour factors

  ! now do the initialization
  call InitDglapHolder(grid, dglap_h, factscheme, nloop)
\end{verbatim}
The constants \ttt{factscheme\_*} are defined in \ttt{module
  dglap\_choices}. %
The corrections to the splitting functions to get the DIS scheme are
implemented by carrying out appropriate convolutions of the $\MSbar$
splitting and coefficient functions. Currently the DIS scheme is only
implemented to NLO. (It's NNLO implementation would actually be fairly
straightforward given the parameterizations provided in
\cite{White:2005wm}) The polarised splitting functions are only known
to NLO.

Initialization can also be carried out with a single call go for a range of
different numbers of flavours:
\begin{verbatim}
  integer :: nflo, nfhi
  [...]
  nflo = 3; nfhi = 6    ! [calls to qcd_SetNf handled automatically]
  call InitDglapHolder(grid, dglap_h, factscheme, nloop, nflo, nfhi)
\end{verbatim}
Mass thresholds are not currently correctly supported in the DIS
scheme.

For all the above calls, at NNLO the choice of exact of parameterized
splitting functions and mass thresholds is determined by the calls to
\ttt{dglap\_Set\_nnlo\_splitting} and
\ttt{dglap\_Set\_nnlo\_nfthreshold}, as described in
sections~\ref{sec:dglap-split}, \ref{sec:mtm} respectively. These
calls must be made prior to the call to \ttt{InitDglapHolder}.

Having initialised a \ttt{dglap\_holder} one has access to various components:
\begin{verbatim}
  type dglap_holder
    type(split_mat), pointer :: allP(1:nloop, nflo:nfhi)  ! FFNS: nflo=nfhi=nf_lcl
    type(split_mat), pointer :: P_LO, P_NLO, P_NNLO
    type(mass_threshold_mat) :: MTM2
    logical                  :: MTM2_exists
    integer                  :: factscheme, nloop
    integer                  :: nf
    [ ... ]
  end type dglap holder
\end{verbatim}
Some just record information passed on initialisation, for example
\ttt{factscheme} and \ttt{nloop}. Other parts are set up once and for
all on initialisation, notably the \ttt{allP} matrix, which contains
the 1-loop, 2-loop, etc. splitting matrices for the requested $n_f$
range.

Yet other parts of the \ttt{dglap\_holder} type depend on $n_f$.
Before accessing these, one should first perform the following call:
\begin{verbatim}
  call SetNfDglapHolder(dh, nf_lcl)
\end{verbatim}
This creates links:
\begin{verbatim}
  dh%P_LO   => dh%allP(1,nf_lcl)
  dh%P_NLO  => dh%allP(2,nf_lcl)
  dh%P_NNLO => dh%allP(3,nf_lcl)
\end{verbatim}
for convenient named access to the various splitting matrices, and it
also sets the global (\ttt{qcd}) $n_f$ value (via a call to
\ttt{qcd\_SetNf}) and where relevant updates the internal $n_f$ value
associated with \ttt{MTM2} (via a call to \ttt{SetNfMTM}).

As with other types that allocate memory for derived types, that
memory can be freed via a call to the \ttt{Delete} subroutine,
\begin{verbatim}
  call Delete(dh)
\end{verbatim}

%======================================================================
%======================================================================
\section{Renormalisation Group Evolution}

\comment{We slowly approach our objective}


%----------------------------------------------------------------------
\subsection{Running coupling}
\label{sec:run-coupl}

Before carrying out any DGLAP evolutions, one first needs to set up a
a \ttt{running\_coupling} object (defined in \ttt{module
  qcd\_coupling}):
\begin{verbatim}
  type(running_coupling) :: coupling
  real(dp) :: alfas, Q, quark_masses(4:6), muMatch_mQuark
  integer  :: nloop, fixnf

  [... set parameters ...]
  call InitRunningCoupling(coupling [, alfas] [, Q] [, nloop] [, fixnf]&
                          & [, quark_masses] [, muMatch_mQuark])
\end{verbatim}
As can be seen, many of the arguments are optional. Their default
values are as follows:
\begin{verbatim}
  alfas = 0.118_dp
  Q     = 91.2_dp
  nloop = 2
  fixnf = [.not. present]
                       !     charm,      bottom,   top
  quark_masses(4:6) = (/ 1.414213563_dp, 4.5_dp, 175.0_dp /)
  muMatch_mQuark    = 1.0_dp
\end{verbatim}
The running coupling object is initialised so that at scale \ttt{Q}
the coupling is equal to \ttt{alfas}. The running is carried out with
the \ttt{nloop} $\beta$-function. If the \ttt{fixnf} argument is
present, then the number of flavours is kept fixed at that
value. Otherwise flavour thresholds are implemented at scales
\begin{verbatim}
  muMatch_mQuark * quark_masses(4:6)
\end{verbatim}
where the quark masses are \emph{pole} masses. The choice to use pole
masses (and their particular default values) is inspired by the
benchmark runs~\cite{Benchmarks} in which \hoppet results were
compared to those of Vogt's moment-space code
QCD-Pegasus~\cite{Pegasus}.

To access the coupling at some scale \ttt{Q} one uses the following
function call:
\begin{verbatim}
  alfas = Value(coupling, Q [, fixnf])
\end{verbatim}
This is the value of the coupling as obtained from the Runge-Kutta
solution of the \ttt{nloop} version of eq.(\ref{eq:as-ev}) (the
evolution is actually carried out for $1/\as$), together with the
appropriate mass thresholds \cite{coupling_mass_thresholds}. For
typical values of $\as(M_Z)$ the coupling is guaranteed to be reliably
determined in the range $0.5 \GeV < Q < 10^{19}\GeV$. The values of
the $\beta$ function coefficients used in the evolution correspond to
those obtained with the values of the QCD colour factors that were in
vigor at the moment of initialisation of the coupling.

In the variable flavour-number case, the \ttt{fixnf} argument allows
one to obtain the coupling for \ttt{fixnf} flavour even outside the
natural range of scales for that number of flavours. This is only
really intended to be used close to the natural range of scales, and
can be slow if one goes far from that range (a warning message will be
output). If one is interested in a coupling that (say) never has more
than 5 active flavours, then rather than using the \ttt{fixnf} option
in the \ttt{Value} subroutine, it is best to initialize the coupling
with a fictitious large value for the top mass.

Often it is convenient to be able to enquire about the mass information
embodied in a \ttt{running\_coupling}. For example in the PDF
evolution below, all information about the location of mass thresholds is
obtained from the \ttt{running\_coupling} type.

The quark pole mass for flavour \ttt{iflv} can be obtained with the
call
\begin{verbatim}
  pole_mass = QuarkMass(coupling, iflv)
\end{verbatim}
The range of scales, $\ttt{Qlo} < Q < \ttt{Qhi}$ for which \ttt{iflv}
is the heaviest active flavour is obtained by the subroutine call
\begin{verbatim}
  call QRangeAtNf(coupling, iflv, Qlo, Qhi [, muM_mQ])
\end{verbatim}
The optional argument \ttt{muM\_mQ} allows one to obtain the answer as
if one had initialised the coupling with a different value of
\ttt{muMatch\_mQuark} than that actually used. One can also establish
the number of active flavours, \ttt{nf\_active}, at a given scale
\ttt{Q} with the following function:
\begin{verbatim}
  nf_active = NfAtQ(coupling, Q [, Qlo, Qhi] [, muM_mQ])
\end{verbatim}
As well as returning the number of active flavours, it can also set
\ttt{Qlo} and \ttt{Qhi}, which correspond to the range of scales in
which the number of active flavours is unchanged. The optional
\ttt{muM\_mQ} argument has the same purpose as in the \ttt{QRangeAtNf}
subroutine. The last of the enquiry functions allows one to obtain the
range of number of flavours covered in this coupling, $\ttt{nflo} \le
n_f \le \ttt{nfhi}$:
\begin{verbatim}
  call NfRange(coupling, nflo, nfhi)
\end{verbatim}

Finally, as usual, once you no longer need a \ttt{running\_coupling}
object, you may free the memory associated with it using the
\ttt{Delete} call:
\begin{verbatim}
  call Delete(coupling)
\end{verbatim}



%----------------------------------------------------------------------
\subsection{DGLAP evolution}
\label{sec:dglap-ev}


%......................................................................
\subsubsection{Direct evolution}
\label{sec:direct-evolution}

We are now ready to evolve a PDF. This is done by breaking the
evolution into steps, and for each one using a Runge-Kutta
approximation for the solution of a first-order matrix differential
equation. The steps are of uniform size in a variable $u$ that
satisfies the following approximate relation
\begin{equation}
  \label{eq:du}
  \frac{du}{d\ln Q^2} \simeq \as(Q^2)
\end{equation}
For a 1-loop running coupling one has $u = (\ln \ln
Q^2/\Lambda)/\beta_0$, which is the variable that appears in
analytical solutions to the $1$-loop DGLAP equation.
%
The step size in u, du, can be set with the following call
\begin{verbatim}
  real(dp) :: du = 0.1_dp ! or some smaller value
  call SetDefaultEvolutionDu(du)
\end{verbatim}
The error on the evolution from the finite step size should scale as
$(\ttt{du})^4$. With the default value of $\ttt{du}=0.1$, errors are
typically somewhat smaller than $10^{-3}$ (see
section~\ref{sec:benchmarks} for the detailed benchmarks).

To actually carry out the evolution, one uses the following subroutine
call:
\begin{verbatim}
  call EvolvePDF(dh, pdf, coupling, Q_init, Q_end &
                        & [, muR_Q] [, nloop] [, untie_nf] [, du] )
\end{verbatim}
which takes a \ttt{pdf} uses the splitting matrices in \ttt{dh}, to
evolve it from scale \ttt{Q\_init} to scale \ttt{Q\_end}
%
By default the renormalisation to factorisation scale ratio is
$\ttt{muR\_Q} = 1.0$ and the number of loops in the evolution is the
same as was used for the running \ttt{coupling} (the \ttt{nloop}
optional argument makes it possible to override this choice). Variable
flavour-number switching takes place at the pole masses (maybe one day
this restriction will be lifted) as associated with the
\ttt{coupling}. 

If the \ttt{dglap\_holder} object \ttt{dh} does not support the
relevant number of loops or flavours, the program will give an error
message and stop. With the \ttt{untie\_nf} option you can request that
the number of flavours in the evolution be `untied' from that in the
coupling in the regions where \ttt{dh} does not support the number of
flavours used in the coupling. Instead the closest number of flavours
will be used.\footnote{For example if \ttt{dh} was initialised with
  $n_f = 3\ldots5$ while the coupling has $n_f = 3\ldots 6$, then
  variable flavour number evolution will be used up to $n_f = 5$, but
  beyond the top mass the evolution will carry on with $5$ flavours,
  while the coupling uses $6$ flavours. There probably aren't too many
  good reasons for doing this (other than for examining how much it
  differs from a `proper' procedure).}

Mass thresholds (NNLO) are implemented as
\begin{subequations}
\label{eq:mass_threshold}
\begin{align}
  \ttt{pdf}_{n_f} &= \ttt{pdf}_{n_f-1} +
  \left(\frac{\alpha_{s,n_f}(x_\mu M)}{2\pi}\right)^2
  (\mbox{\ttt{dh\%MTM2 .conv.}}\; \ttt{pdf}_{n_f-1})\\
  %
  \ttt{pdf}_{n_f-1} &= \ttt{pdf}_{n_f} \;\;\;\;+
  \left(\frac{\alpha_{s,n_f}(x_\mu M)}{2\pi}\right)^2
  (\mbox{\ttt{dh\%MTM2 .conv.}}\; \ttt{pdf}_{n_f})
\end{align}
\end{subequations}
when crossing the threshold upwards and downwards, respectively. Note
that the two operations are not perfect inverses of each other,
because the number of flavours of the \ttt{pdf} used in the
convolution differs in the two cases. The mismatch however is only of
order $\as^4$ (NNNNLO), \ie well beyond currently known accuracies.

A general remark is that crossing a flavour threshold downwards will
nearly always result in some (physically spurious?) intrinsic
heavy-flavour being left over below threshold.

%......................................................................
\subsubsection{Precomputed evolution and the \ttt{evln\_operator}}
\label{sec:precomputed-evolution}

Each Runge-Kutta evolution step involves multiple evaluations of the
derivative of the PDF, and the evolution between two scales may be
broken up into multiple Runge-Kutta steps. This amounts to a large
number of convolutions. It can therefore be useful to create a single
\emph{derived} splitting matrix that is equivalent to the whole
evolution between the two scales. 

A complication arises because evolutions often cross flavour
thresholds, whereas a derived splitting matrix is only valid for fixed
$n_f$. Therefore a new type has to be created, \ttt{evln\_operator},
which consists of a linked list of splitting and mass threshold
matrices, breaking an evolution into a chain of interleaved
fixed-flavour evolution steps and flavour changing steps. An
\ttt{evln\_operator} is created with a call that is almost identical
to that used to evolve a PDF:
\begin{verbatim}
  type(evln_operator) :: evop
  real(dp), pointer   :: pdf_init(:,:), pdf_end(:,:)
  [...]
  call InitEvlnOperator(dh, evop, coupling, Q_init, Q_end &
                           & [, muR_Q] [, nloop] [, untie_nf] [, du] )
\end{verbatim}
It can then be applied to PDF in the same way that a normal
\ttt{split\_mat} would:
\begin{verbatim}
  pdf_end = evop * pdf_init       ! assume both pdfs already allocated
  ! OR (alternate form) 
  pdf_end = evop .conv. pdf_init  
\end{verbatim}
As usual the \ttt{Delete} subroutine can be used to clean up any
memory associated with an evolution operator that is no longer needed.


%======================================================================
%======================================================================
\section{Tabulated PDFs}
\label{sec:tabulated-pdfs}

The tools in the previous section are useful if one knows that one
needs DGLAP evolution results at a small number of predetermined $Q$
values. Often however one simply wishes to provide a PDF distribution
at some initial scale and then subsequently be able to access it at
arbitrary values of $x$ and $Q$. For this purpose it is useful to
produce a table of the PDF as a function of $Q^2$, which then allows
for access to the PDF at arbitrary $x, Q$ using an interpolation.

Name changes?
\begin{verbatim}
  pdftab                  -> pdf_table
  pdftab_AllocTab         -> AllocPdfTable
  pdftab_AssocNfInfo      -> AddNfInfoToPdfTable

  pdftab_InitTabSub       -> FillPdfTable_f90sub  ???
  pdftab_InitTab_LHAPDF   -> FillPdfTable_LHAPDF  ???

  pdftab_InitTabEvolve    -> EvolvePdfTable()

  pdftab_PreEvolve        -> PreEvolvePdfTable()

  pdftab_ValTab_yQ/xQ     -> EvalPdfTable(...)       
\end{verbatim}




%======================================================================
%======================================================================
\section{Vanilla interface}
\label{sec:vanilla}


%======================================================================
%======================================================================
\section{Benchmarks and timings}
\label{sec:benchmarks}


Comparisons to $N$-space codes (especially Vogt):

\begin{itemize}
\item $N$-space is good for a moderate (few hundred) number of $x$-$Q$
  points (for lots of points, better to convert it to tabulation)
%
\item For $\sim 500$ points, $N$-space is faster for the same
  accuracy, and its running times probably scale much more tamely when
  you try to go to extremely high accuracies.
%
\item Nevertheless, for reference purposes the $x$-space code can also
  go to high accuracies ($10^{-7}$) if you're willing to accept slow
  runs.
%
\item At accuracies that are acceptable for phenomenology ($10^{-3} -
  10^{-4}$) the $x$-space code runs nearly as fast as Vogt's $N$-space
  code in its lower accuracy option (which is still better accuracy
  than the $x$-space code). But overheads are different --- evolution
  is a one-off operation, accessing points is then done using
  interpolation.
%
\item $N$-space isn't an option if you don't have a simple analytic
  form for your initial parton distributions.
%
\item currently no other code has the option of the exact NNLO
  splitting functions.
\end{itemize}

%======================================================================
\appendix

%======================================================================
\section{Useful tips on fortran~95}
\label{sec:f95appendix}

As fortran~95's use in high-energy physics is not as widespread as
that of other languages such as fortran~77 and C++, it is useful to
summarise some key novelties compared to fortran~77, as well as some
points that might otherwise cause confusion. For further information
the reader is referred both to books about the language such as
\cite{F95Explained} and to web resources~\cite{F95WebResources}.

\paragraph{Free form.}  Most of the code in the \hoppet package is in
free-form. The standard extension for free-form form files is
\ttt{.f90}. There is no requirement to leave 6 blank spaces before
every line and lines can consist of up to 132 characters. The other
main difference relative to f77 fixed form is that to continue a line
one must append an append an ampersand, \ttt{\&}, to the line to be
continued. One may optionally include an ampersand as the first
non-space character of the continuation line.

For readibility, many of the subprogram names in this documentation
are written with capitals at the start of each word. Note however that
free-form fortran~95, like its fixed-form predecessors, is case
insensitive.

\paragraph{Modules, and features relating to arrays.} Fortran~90/95
allows one to package variables and subroutines into modules
\begin{verbatim}
module test_module
  implicit none
  integer :: some_integer
contains
  subroutine print_array(array)
    integer, intent(in) :: array(:) ! size is known, first element is 1
                                    ! intent(in) == array will not be changed
    integer             :: i, n
    n = size(array)
    do i = 1, n
      print *, i, array(i)
    end do
  end subroutine hello_world
end module test_module
\end{verbatim}
The variable \texttt{some\_integer} and the subroutine
\texttt{print\_array} are invisible to other routines unless they
explicitly \texttt{use} the module as in the following example:
\begin{verbatim}
program test_program
  use test_module
  implicit none
  integer :: array1(5), array2(-2:2)
  integer :: i
  
  some_integer = 5   ! set the variable in test_module
  array1       = 0   ! set all elements of array1 to zero
  array2(-2:0) = 99  ! set elements 1..3 of array2 to equal to 3.
  array2(1:2)  = 2*array2(-1:0)  ! elements -2..0 equal twice elements -1..0

  print *, "Printing array 1"
  call print_array(array1)
  print *, "Printing array 2"
  call print_array(array2)
end program test_program
\end{verbatim}
Constants can be assigned to arrays (\texttt{array1}) or array
subsections (\texttt{array2(-2:0)}), arrays can be assigned to arrays
of the same size (as is done for \texttt{array2(-2:0)}) and
mathematical operations apply to each element of the array (as with
the multiplication by 2).

When arrays are passed to function or subroutine that is defined in a
\texttt{use}d module, information about the size of the array is
passed along with the array itself. Note however that information
about the lower bound is \emph{not} passed, so that for both
\texttt{array1} and \texttt{array2}, \texttt{print\_array} will see
arrays whose valid indices will run from $1\ldots5$. Thus the output
from the program will be
\begin{verbatim}
 Printing array 1
 1 0
 2 0
 3 0
 4 0
 5 0
 Printing array 2
 1 99
 2 99
 3 99
 4 198
 5 198
\end{verbatim}
If \texttt{print\_array} wants \texttt{array} to have a different lower
bound it must specify it in the declaration, for example
\begin{verbatim}
  integer, intent(in) :: array(-2:) ! size is known, first element is -2
\end{verbatim}
While it may initially seem bizarre, there are good reasons for such
behaviour (for example in allowing a subroutine to manipulate multiple
arrays of the same size without having to worry about whether they all
have the same lower bounds).

\paragraph{Dynamic memory allocation, pointers.} One of the major
additions of f95 compared to f77 is that of dynamic memory allocation,
for example with pointers
\begin{verbatim}
  integer, pointer :: dynamic_array(:)
  allocate(dynamic_array(-6:6))
  ! .. work with it ..
  deallocate(dynamic_array)
\end{verbatim}
This is fundamental to our ability to decide parameters of the PDF
grid(s) at runtime. Pointers can be passed as arguments to subprograms.
If the subprogram does not specify the \texttt{pointer} attribute for
the dummy argument
\begin{verbatim}
subroutine xyz(dummy_array)
  integer, intent(in) :: dummy_array(:)
\end{verbatim}
then everything behaves as if the argument were a normal array (\eg
the default lower bound is $1$). Alternatively the subroutine can
specify that it expects a pointer argument
\begin{verbatim}
subroutine xyz(dummy_pointer_array)
  integer, pointer :: dummy_pointer_array(:)
\end{verbatim}
In this case the subroutine has the freedom to allocate and deallocate
the array. Note also that because a pointer to the full array
information is being passed, the lower bound of \texttt{dummy\_pointer\_array}
is now the same as in the calling routine. Though this sounds like a
technicality, it is important because a corollary it that a subroutine
can allocate a dummy pointer array with bounds that are passed back to
the calling subroutine (we need this for the flavour dimension of
PDFs, whose lower bound is most naturally $-6$). 

Note that in contrast to \ttt{C}/\ttt{C++} pointers, f90 pointers do
not explicitly need to be dereferenced --- in this respect they are
more like \ttt{C++} \emph{references}. To associate a pointer with an
object, one uses the \ttt{=>} syntax:
\begin{verbatim}
  integer, target  :: target_object(10)
  integer, pointer :: pointer_to_object(:)
  pointer_to_object => target_object
  pointer_to_object(1:10) = 0        ! sets target_object(1:10)
\end{verbatim}
One notes that the object that was pointed to had the \ttt{target}
attribute --- this is mandatory (unless the object is itself a
pointer).


\paragraph{Derived types.} Another feature of f95 that has been
heavily used is that of derived types (analogous to C's
\texttt{struct}):
\begin{verbatim}
  type pair
    integer first, second
  end type pair 
\end{verbatim}
Variables of this type can then be created and used as follows
\begin{verbatim}
  type(pair) :: pair_object, another_pair_object
  pair_object%first  = 1
  pair_object%second = 2
  another_pair_object = pair_object
  print *, another_pair_object%second
\end{verbatim}
where one sees that the entirety of the object can be copied with the
assignment (\texttt{=}) operator. Note that many of the derived types
used in \hoppet contain pointers and when such a derived type object
is copied, the copy's pointer just points to the same memory as the
original object's pointer. This is sometimes what you want, but on
other occasions will give unexpected behaviour: for example splitting
function types are derived types containing pointers, so when you
assign one splitting function object to another, they end up referring
to the same memory, so if you multiply one of them by a constant, the
other one will also be modified.

\paragraph{Operator overloading} While assignment behaves more or less
as expected by default with derived types (it can actually be modified
if one wants to), other operators do not have default definitions. So
if one wants to define, say, a multiplication of objects one may
associate a function with a given operator, using an interface block:
\begin{verbatim}
module test_module
  interface operator(*)        ! provide access to dot_pairs through 
    module procedure dot_pairs ! the normal multiplication symbol
  end interface 
  interface operator(.dot.)    ! provide acecss to dot_pairs through
    module procedure dot_pairs ! a specially named operator
  end interface 
contains
  integer function dot_pairs(pair1, pair2)
    type(pair), intent(in) :: pair1, pair2
    dot_pairs = pair1%first*pair2%first + pair1%second*pair2%second
  end function dot_pairs
end module
\end{verbatim}
given which we can then write
\begin{verbatim}
  integer    :: i
  type(pair) :: pair1, pair2
  [... some code to set up pair values ...]
  ! now multiply them
  i = pair1 * pair2
  i = pair1 .dot. pair2  ! equivalent to previous statement
\end{verbatim}
Since the the multiplication operator (\texttt{*}) already exists for
all the default types, by defining it for a new type we have
\emph{overloaded} it. Note that there are some subtleties with
precedences of user-defined operators: operators (like \texttt{*})
that already exist have the same precedence as they have is usual
operators; operators that do not exist by default (\texttt{.dot}) have
the lowest possible preference, so, given the above definitions,
\begin{verbatim}
  i = 2 + pair1 * pair2       ! legal
  i = 2 + pair1 .dot. pair2   ! illegal, means: (2+pair1).dot.pair2
  i = 2 + (pair1 .dot. pair2) ! legal
\end{verbatim}
where the second line is illegal because we have not defined any
operator for adding an integer and a pair. Similarly care is needed
when using the \hoppet's operator \texttt{.conv.}.

\paragraph{Floating point precision:}
A final point concerns floating point variable types. Throughout we
have used definitions such as
\begin{verbatim}
  real(dp), pointer :: pdf(:,:)
\end{verbatim}
and written numbers with a trailing \texttt{\_dp}
\begin{verbatim}
  param = 1.7_dp
\end{verbatim}
Here \texttt{dp} is an integer parameter (defined in the
\texttt{types} module and accessible also through the
\texttt{hoppet\_v1} module), which specifies the \texttt{kind} of real
that we want to define, specifically double precision. We could also
have written \texttt{double precision} everywhere, but this is less
compact, and the use of a kind parameter has the advantage that we
can just modify its definition in one point in the program and the
precision will be modified everywhere. (Well, almost, since some
special functions are written in fortran~77 using \texttt{double
  precision} declarations and do their numerics based on the
assumption that that truly is the type they're dealing with).


\paragraph{Optional and keyword arguments}


%======================================================================
\begin{thebibliography}{99}

% about 1 minute at NLO.
\bibitem{coriano} A.~Cafarella and C.~Coriano,
%``Direct solution of renormalization group equations of QCD in x-space: NLO
%implementations at leading twist,''
Comput.\ Phys.\ Commun.\  {\bf 160} (2004) 213
[arXiv:hep-ph/0311313].
%%CITATION = HEP-PH 0311313;%%

% Uses decomposition on Laguerre polynomials -- about
% 30 of them, remains Y^2 * T method. Initialisation
% (transform of splitting functions takes 15s on thalie)
% (didn't try evolution; didn't check accuracy; evolution
% times and accuracy are not mentioned; seemed fixed nf)
\bibitem{Schoeffel:1998tz}
L.~Schoeffel,
%``An elegant and fast method to solve QCD evolution equations,  application to
%the determination of the gluon content of the pomeron,''
Nucl.\ Instrum.\ Meth.\ A {\bf 423} (1999) 439.
%%CITATION = NUIMA,A423,439;%%
See also \url{http://www.desy.de/~schoffel/L_qcd98.html},
\url{http://www-spht.cea.fr/pisp/gelis/Soft/DGLAP/index.html}


\bibitem{Weinzierl:2002mv}
S.~Weinzierl,
%``Fast evolution of parton distributions,''
Comput.\ Phys.\ Commun.\  {\bf 148} (2002) 314
[arXiv:hep-ph/0203112];
%%CITATION = HEP-PH 0203112;%%
%\bibitem{Roth:2004ti}
M.~Roth and S.~Weinzierl,
%``QED corrections to the evolution of parton distributions,''
Phys.\ Lett.\ B {\bf 590} (2004) 190
[arXiv:hep-ph/0403200].
%%CITATION = HEP-PH 0403200;%%



\bibitem{Pascaud:2001bi}
C.~Pascaud and F.~Zomer,
%``A fast and precise method to solve the Altarelli-Parisi equations in x
%space,''
arXiv:hep-ph/0104013.
%%CITATION = HEP-PH 0104013;%%

\bibitem{F95Explained}
  M. Metcalf and J. Reid, \emph{Fortran 90/95 Explained}, Oxford
  University Press, 1996.

\bibitem{F95WebResources} Many introductions and tutorials about
  fortran~90 may be found at
  \url{http://dmoz.org/Computers/Programming/Languages/Fortran/Tutorials/Fortran_90_and_95/}

\bibitem{vanNeerven:1999ca}
  W.~L.~van Neerven and A.~Vogt,
  %``NNLO evolution of deep-inelastic structure functions: The non-singlet
  %case,''
  Nucl.\ Phys.\ B {\bf 568} (2000) 263
  [arXiv:hep-ph/9907472].
  %%CITATION = HEP-PH 9907472;%%

\bibitem{vanNeerven:2000uj}
  W.~L.~van Neerven and A.~Vogt,
  %``NNLO evolution of deep-inelastic structure functions: The singlet case,''
  Nucl.\ Phys.\ B {\bf 588} (2000) 345
  [arXiv:hep-ph/0006154].
  %%CITATION = HEP-PH 0006154;%%

\bibitem{NRf90}
  Press {\it et al.}, \emph{Numerical Recipes in Fortran~90},
  Cambridge University Press, 1996.

\bibitem{LHAPDF} W.~Giele and M.~R.~Whalley,
\url{http://hepforge.cedar.ac.uk/lhapdf/}


%\cite{Moch:2004pa}
\bibitem{NNLO-NS}
  S.~Moch, J.~A.~M.~Vermaseren and A.~Vogt,
  %``The three-loop splitting functions in QCD: The non-singlet case,''
  Nucl.\ Phys.\ B {\bf 688} (2004) 101
  [arXiv:hep-ph/0403192].
  %%CITATION = HEP-PH 0403192;%%

%\cite{Vogt:2004mw}
\bibitem{NNLO-singlet}
  A.~Vogt, S.~Moch and J.~A.~M.~Vermaseren,
  %``The three-loop splitting functions in QCD: The singlet case,''
  Nucl.\ Phys.\ B {\bf 691} (2004) 129
  [arXiv:hep-ph/0404111].
  %%CITATION = HEP-PH 0404111;%%


\bibitem{FortranPolyLog}
  T.~Gehrmann and E.~Remiddi,
  %``Numerical evaluation of two-dimensional harmonic polylogarithms,''
  Comput.\ Phys.\ Commun.\  {\bf 144} (2002) 200.
  %[hep-ph/0111255].
  %%CITATION = HEP-PH 0111255;%%


\bibitem{NNLO-MTM}
  M.~Buza, Y.~Matiounine, J.~Smith, R.~Migneron and W.~L.~van Neerven,
  %``Heavy quark coefficient functions at asymptotic values $Q~2 \gg m~2$,''
  Nucl.\ Phys.\ B {\bf 472}, 611 (1996)
  [arXiv:hep-ph/9601302];\\
  %%CITATION = HEP-PH 9601302;%%
%
  M.~Buza, Y.~Matiounine, J.~Smith and W.~L.~van Neerven,
  %``Charm electroproduction viewed in the variable-flavour number scheme
  %versus fixed-order perturbation theory,''
  Eur.\ Phys.\ J.\ C {\bf 1}, 301 (1998)
  [arXiv:hep-ph/9612398].
  %%CITATION = HEP-PH 9612398;%%
  
\bibitem{VogtMTMParam} A.~Vogt, private communication.


\bibitem{White:2005wm}
  C.~D.~White and R.~S.~Thorne,
  %``Comparison of NNLO DIS scheme splitting functions with results from exact
  %gluon kinematics at small x,''
  Eur.\ Phys.\ J.\ C {\bf 45} (2006) 179
  [arXiv:hep-ph/0507244].
  %%CITATION = HEP-PH 0507244;%%

\bibitem{Benchmarks} 
  W.~Giele {\it et al.},
  ``Les Houches 2001, the QCD/SM working group: Summary report,''
  hep-ph/0204316, section 1.3;\\
  %%CITATION = HEP-PH 0204316;%%
  M.~Dittmar {\it et al.},
  ``Parton distributions: Summary report for the HERA-LHC workshop,''
  hep-ph/0511119, section 4.4.
  %%CITATION = HEP-PH 0511119;%%


\bibitem{Pegasus}
  A.~Vogt,
  %``Efficient evolution of unpolarized and polarized parton distributions  with
  %QCD-PEGASUS,''
  Comput.\ Phys.\ Commun.\  {\bf 170} (2005) 65
  [arXiv:hep-ph/0408244].
  %%CITATION = HEP-PH 0408244;%%

\end{thebibliography}
\end{document}
